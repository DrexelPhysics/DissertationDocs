\section{Methods}

\subparagraph*{}\label{sec:I27}
The experiments were carried out on octomers of I27 (\cref{fig:I27}).
\index{I27}
I27 is a model protein that has been used in mechanical unfolding
experiments since the first use of synthetic
chains\citep{carrion-vazquez99b,TODO}.  It was used here because it is
both well characterized and readily available (%
\href{http://www.athenaes.com/}{AthenaES}, Baltimore, MD,
\href{http://www.athenaes.com/I27OAFMReferenceProtein.php}{0304}).

I27's unfolding mechanism seems to involve stretching into a metastable
intermediate state followed by Bell-model escape to the unfolded
state\citep{marszalek99}, although there is not yet a consensus of
the presense of the proposed intermediate\citep{TODO}.

The I27 octamers were stored in a TODO buffer solution.

Mechanical unfolding experiments were carried out on I27 octomers
(AthenaES) in PBS on gold-coated coverslips.  We used both cantilevers
on Olympus's OMCL-TR400-PSA-1, which are nominally $80$ and
$20\U{pN/nm}$.  Promising sawtooth curves were selected by eye and fit
to WLCs\index{WLC} to identify I27 unfolding events.  The results were
sorted into two bins according to cantilever stiffness, and then
averaged across each cantilever-stiffness/pulling-speed group to
produce \cref{fig:cant:v-dep}.

\begin{figure}
  \asyinclude{figures/cantilever-data/v-dep}
  \caption{Pulling speed dependence of I27 for different cantilever
    stiffnesses.  The listed stiffnesses are averages across several
    individual cantilevers and calibrations.  Each box is the average
    of some number of unfolding events, and the box area is
    proportional to that number.  There are $82$ unfolding events for
    the stiff cantilevers and $274$ for the soft cantilevers.%
    \label{fig:cant:v-dep}}
\end{figure}

Unfortunately, the data are not of high enough quality to extract the
unfolding parameters $k$ or $\Delta x$.  Note that the increase in
mean unfolding force is not entirely due to the increased loading rate
of the stiffer cantilever, because the difference is still present in
the loading rate dependence (\cref{fig:cant:load-dep}).  The
loading rates were extracted from the data by taking the slope of the
fit WLC\index{WLC} at unfolding.

\begin{figure}
  \asyinclude{figures/cantilever-data/loading-rate}
  \caption{Loading rate.\label{fig:cant:load-dep}}
\end{figure}
