\section{Contour integration}

As a brief review, some definite integrals from $-\infty$ to $\infty$
can be evaluated by integrating along the contour \C\ shown in
\cref{fig:UHP-contour}.
%
\nomenclature[o  ]{$\infty$}{Infinity}

\begin{figure}
  \asyinclude{figures/contour/contour}
  \caption{Integral contour \C\ enclosing the upper half of the
    complex plane.  If the integrand $f(z)$ goes to zero ``quickly
    enough'' as the radius of \C\ approaches infinity, then the only
    contribution comes from integration along the real axis (see text
    for details).\label{fig:UHP-contour}}
\end{figure}

A sufficient condition on the function $f(z)$ to be integrated, is
that $\lim_{\abs{z}\rightarrow\infty}\abs{f(z)}$ falls off at least as
fast as $\frac{1}{z^2}$.  When this is the case, the integral around
the outer semicircle of \C\ is 0, so the
$\iC{f(z)}=\iInfInf{z}{f(z)}$.

We can evaluate the integral using the residue theorem\index{residue theorem},
\begin{equation}
  \iC{f(x)} = \sum_{z_p \in \{\text{poles in \C}\}} 2\pi i \Res{z_p}{f(z)} \;,
                                        \label{eq:res-thm}
\end{equation}
where for simple poles (single roots)
\begin{equation}
  \Res{z_p}{f(z)} = \limZp(z-z_p) f(z) \;, \label{eq:res-simple}
\end{equation}
and in general for a pole of order $n$
\begin{equation}
  \Res{z_p}{f(z)} = \frac{1}{(n-1)!} \cdot\limZp
                       \nderiv{n-1}{z}{}\left[ (z-z_p)^n \cdot f(z) \right] \;.
                                        \label{eq:res-general}
\end{equation}

%We also use the following theorem
%\begin{thm} \label{thm:res-even-fn}
%If $f(z) = f(-z)$, then $\Res{z_p}{f(z)} = \Res{-z_p}{f(z)}$.\\
%\end{thm}
%\begin{proof}
%If $g(z)$ is even, then $\deriv{z}{g}$ must be odd
% (consider the Lorentz expansion).
%By induction, $\nderiv{n}{z}{g}$ is even when $n$ is even,
%and odd when $n$ is odd.
%By inspection, $(z-z_p)^n$ is also even iff $n$ is even.
%Therefore, 
%\begin{equation}
%  (z-z_p)^n \cdot f(z)
%\end{equation}
%is even iff $n$ is even, so
%\begin{equation}
%  \nderiv{n-1}{z}{}\left( (z-z_p)^n \cdot f(z) \right)
%\end{equation}
%is always even, so 
%\begin{equation}
%  \Res{z_p}{f(z)} = \frac{1}{(n-1)!} \cdot\limZp
%                       \nderiv{n-1}{z}{}\left[ (z-z_p)^n \cdot f(z) \right]
%\end{equation}
%must be even with respect to $z_p$ $\qed$.
%\end{proof}
