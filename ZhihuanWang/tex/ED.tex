\chapter{Reduced Electronic Density Distribution} \label{ED}

\section{Self-consistent Poisson-Schrodinger Solver} \label{sec:model}

%\subsection{What Reddening Law Fits Better?}\label{sec:DIC}


\section{Electronic Distribution in Nanowires} \label{sec:spectra}
Figure 8 (C) shows the FDTD-simulated electric field density of a hexagonal nanowire at y cross section (top) and x cross section (bottom). The photon energy of this mode shown as the insets of Fig. 8(C) is concentrated primarily along the 6 corners and secondarily along the facets with little light in the 3D core of GaAs. Hence, we suggest that the fortuitous spatial overlap of the resonant optical modes on reduced dimensional electronic wavefunctions plays a significant role in the remarkable optoelectronic properties of core-shell nanowires. Restated, the superposition of the photon modes  on reduced electronic states that form on the facets and vortices of the hexagonal CSNWs strongly enhances both upward and downward transition rates.  Thus, the reduced dimensionality transition rate distinguishes the core-shell nanostructure from the optically equivalent structures of Fig. 6 due to its significantly modified rate management. These nanostructures are not only excellent optical cavities, but despite their large size also provide the right reduced dimensional electronic structures which enhance optoelectronic interactions.  It should be noted the present analysis is for direct optical transitions; although it can be extended to incorporate k-vector changes as in phonon scattering, other important factors such as many-body interactions need to be included in a comprehensive analysis.

\subsection{Cylindrical Core-Shell Nanowire}

\subsection{Hexagonal Core-shell Nanowire} \label{sec:indv_lines}
\section{Conclusions} \label{sec:conclusions}

