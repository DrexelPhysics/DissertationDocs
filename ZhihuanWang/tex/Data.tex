\chapter{Optical Enhancement of Core-Shell Nanowire} \label{data}


\section{Growth of Nanowires}



\section{Scanning Electron Microscopy Images} 

Figure 1a is top view SEM image of nanowires of ~100nm diameter core of GaAs,
and ~40nm thick AlGaAs, with the inset showing a magnified image that
demonstrates the rather sparse distribution of the wires. Figure 1b shows the
reflectivity of a GaAs wafer on which 50nm thin film of AlGaAs is grown, and
compares this to the reflectivity spectrum of a Si substrate. As expected,
about 30% to 55% of a normally incident light is reflected in bulk Si and GaAs,
with a sharp change for wavelengths near their respective band gaps. Figure 1c
contrasts this with the measured absorption spectrum of two types of GaAs core,
AlGaAs shell (CS) nanowires (BW): those grown on a GaAs substrate (black), and
the others heteroepitaxially grown on a Si substrate (red). The spectra show
that both cases have the signature change of reflectivity at bandgap of GaAs,
i.e., these are due to the GaAs/AlGaAs CSNWs, not the substrate. Importantly,
for the wavelength range of 700-1200nm these core-shells which only occupy ~15%
of the volume compared to thin films of the same height, reflect 2-4% of light
for the CSNWs grown on Si, and 3-7% of light for those grown on GaAs substrate.
The beam-width of the incident light being ~1μm, this shows that only a few NWs
are interrogated by light and, normalized to volume, these wires absorb more
than two orders of magnitude more light than their thin-film counterparts.

\section{Electrical Characterization of Nanowire}

\section{Absorption Enhancement} \label{X-ray}


\section{Emission Enhancement} \label{Dust_data}


\section{Lasing} \label{BH_data}

