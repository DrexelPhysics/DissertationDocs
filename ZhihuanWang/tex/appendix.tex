\chapter{Transition Rates} \label{ch:rates} 

This research project focuses on the dimensional dependence of the absorption
behavior in semiconductor when interacting with light. Through time-dependent
perturbation theory, we find out the transition rate and Fermi’s Golden Rule,
then based on the light interaction Hamiltonian, time average Poynting vector
and matrix element, derive the absorption coefficient for bulk semiconductor
(3-dimension), quantum well (2-dimension) and quantum wire (1-dimension)
structure.  Finally, this report will discuss the partial confinement on the
electron in the conduction band without the hole confined in the valence band
situation.

The Schrödinger equation:
\begin{equation}
  \HAM\Psi(r,t) = - \frac{\hbar}{i}\\frac{\partial }{\partial t}\Psi(r,t)
\end{equation}

The Hamiltonian $\HAM$ can be expressed as:
\begin{equation}
  \HAM = \HAM_0 + \HAM^\prime(r,t)
\end{equation}

where $\HAM_0$ is the unperturbed part Hamiltonian and is time-independent, $\HAM^\prime(r,t)$ is the
small perturbation.

The solution to the unperturbed part is assumed known:

\begin{eqnarray}
\begin{aligned}
  & \HAM_0\Psi_n(r,t) = - \frac{\hbar}{i}\\frac{\partial }{\partial t}\Psi_n(r,t),
  \\
  & \frac{dN_p}{dt} = {\Gamma}v_g{g}N_p + \Gamma\beta_{sp}R_{sp} - \frac{N_p}{\tau_p},
\end{aligned}
\label{eq:eight}
\end{eqnarray}

The time-dependent perturbation is assumed to have the form

\begin{equation}
  \HAM = \begin{cases}
    \HAM^\prime(r)e^{-i\omega t} + \HAM^{\prime+}(r)e^{+\omega t}, & t \geq 0 \\
    0, & t < 0
  \end{cases}
\end{equation}

Expand the wave function in terms of the unperturbed solution, we find out $\Psi(r,t)$:

\begin{equation}
  \Psi(r,t) = \sum_{n}a_n(t)\Phi_n(r)e^{(-iE_{n}t/\hbar)}
\end{equation}

$|a_n(t)|^2$ gives the probability that the electron is in the state n at time t.

Substituting the expansion for $\Psi$ into Schrödinger equation and using (II.3), we
have

\begin{equation}
  \sum_n \frac{da_n(t)}{dt}\Psi_n(r)e^{-iE_{n}t/\hbar} = -\frac{\hbar}{i}\sum_n \HAM^{\prime}(r,t)a_n(t)\Phi_n(r)e^{(-iE_{n}t/\hbar)}
\end{equation}

Taking the inner product with the wave function ${\Phi_m}^\star(r)$, and using the orthonormal
property,

\begin{equation}
  {I_{hm}^{en}}=\int_{-\infty}^\infty{\Phi_n^\ast}(z){g_m(z)}\,\mathrm{d}z,
\end{equation}

\begin{equation}
{I_{hm}^{en}}=\int_{-\infty}^\infty{\Phi_n^\ast}(z){g_m(z)}\,\mathrm{d}z,
\label{eq:seven}
\end{equation}

Therefore, the electron stays at the state i in the absence of any
perturbation. The first order solution is 

If we consider the photon energy to be near resonance, either  or, we find the
dominant terms: The absorption coefficient is a strong function of
dimensionality. The confining situation of the quantum structure will
considerably affect the energy levels, the overlap function of the conduction
and valance band envelope function and also the joint optical density of state.
Through the derivation of the absorption coefficient, we can observe the
interaction between the light and semiconductor.  The next questions should
address: 1. The simulation data for single nanowire abs

\chapter{Semiconductor Laser Modeling}
\label{sec:model}

In this section, we are trying to delve into the mechanics of how an injected
current actually results in an optical output in a semiconductor heterojunction
laser by providing a systematic derivation of the dc light-current
characteristics. First, the rate equation for photon generation and loss in a
laser cavity is developed. This shows that only a small portion of the
spontaneously generated light contributes to the lasing mode. Most of it comes
from the stimulated recombination of carriers. All of the carriers that are
stimulated to recombine by light in a certain mode contribute more photons to
that same mode. Thus, the stimulated carrier recombination/photon generation
process is a gain process. We find the threshold gain for lasing which is the
gain necessary to compensate for cavity losses. The threshold current is the
current required to reach this gain 

For electrons and holes in the active region of a diode laser, only a fraction,
$\eta_i$, of injected current will contribute to the generation of carriers.We
assumed the active regions that are undoped or lightly doped, so that under
high injection levels, charge neutrality applies and the electron density
equals the hole density (i.e., $N = P$ in the active region). Thus, we can
greatly simplify our analysis by specifically tracking only the electron
density, N.

We start with the governing equations of carrier density and photon density in
the active region which is governed by a dynamic process.

\begin{eqnarray}
\begin{aligned}
  & \frac{dN}{dt} = \frac{\eta_{i}I}{qV} - \frac{N}{\tau} - R_{st},
  \\
  & \frac{dN_p}{dt} = {\Gamma}v_g{g}N_p + \Gamma\beta_{sp}R_{sp} - \frac{N_p}{\tau_p},
\end{aligned}
\label{eq:eight}
\end{eqnarray}

where $\beta_{sp}$ is the spontaneous emission factor, defined as the
percentage of the total spontaneous emission coupled into the lasing mode. And it is
just the reciprocal of the number of available optical modes in the bandwidth of
the spontaneous emission for uniform coupling to all modes. The g is Incremental gain
per unit length.

The first term of equation 1 is the rate of injected electrons $G_{gen} =
{\Gamma_{i}I}/{qV}$, ${\Gamma_{i}I}/{q}$ is the number of electrons per
second being injected into the active region, where V is the volume of the
active region. The rest terms are the rate of recombining of electrons per unit
volume in the active region. There are several mechanisms should be considered,
including a spontaneous recombination rate, $R_{sp}$, a nonradiative
recombination rate, $R_{nr}$, a carrier leakage rate, $R_l$ and a net
stimulated recombination, $R_{st}$, including both stimulated absorption and
emission. Which looks like:

\begin{equation}
  R_{rec} = R_{sp} + R_{nr} + R_{l} + R_{st}
\end{equation}

The first three terms on the right refer to the natural or unstimulated carrier
decay processes. The fourth one, $R_{st}$, require the presence of photon.

The natural decay process can be described by a carrier lifetime, $\tau$. In
the absence of photon or a generation term, the rate equation for carrier decay
is $dN/dt = -N/\tau$, where $N/\tau = R_{sp} + R_{nr} + R_{l}$.

The natural decay rate can also be expressed in a power series of the carrier
density, N. We can also rewrite $R_{rec} = BN^2 + (AN + CN^3) + R_{st}$.  Where
$R_{sp} ~ BN^2$ and $R_{nr} + R_{l} ~ (AN + cN^3)$. The coefficient B is the
bimolecular recombination coefficient, and it has a magnitude, $B \sim 10^{-10}
cm^3/s$ for most AlGaAs and InGaAsP alloys of interest.

When a laser is below threshold, in which the gain is insufficient to
compensate for cavity losses, the generated photons do not receive net
amplification. The spontaneous photon generation rate per unit volume is
exactly equal to the spontaneous electron recombination rate, $R_{sp}$, because
an electron-hole pair will emit a photon when they recombine radiatively.
Under steady-state conditions ($dN/dt = 0$), the generation rate equals the
recombination rate with $R_{st} = 0$.

\begin{equation}
  \frac{\eta_{i}I}{qV} = R_{sp} + R_{nr} + R_{l}
\end{equation}

The spontaneously generated optical power, $P_{sp}$, is obtained by multiplying
the number of photons generated per unit time per unit volume, $R_{sp}$, by the
energy per photon, $hv$, and the volume of the active region, V.

\begin{equation}
  P_{sp} = h{\upsilon}VR_{sp} = \eta_i\eta_r\frac{h\upsilon}{q}I
\end{equation}

The main photon generation term above threshold is $R_{st}$. Electron-hole pair
is stimulated to recombine, another photon is generated. But since the cavity
volume occupied by photons, $V_p$, is usually larger than the active region
volume occupied by electrons, V, the photon density generation rate will be
$[V/V_p]R_{st}$, not just $R_{st}$. The electron-photon overlap factor,
$[V/V_p]$, is generally referred to as the confinement factor, $\Gamma$.

\section{Threshold or Steady-State Gain in Lasers} \label{sec:conditionals}

The cavity loss can be characterized by a photon decay constant or lifetime,
$\tau_p$, and the gain necessary to overcome losses, and thus reach threshold.
By assuming steady-state conditions (\ie $dN_p/dt = 0$), and solving for this
steady-state or threshold gain, $g_{th}$, where the generation term equals the
recombination term for photons. We assume only a small fraction of the
spontaneous emission is coupled into the mode (\ie $\beta_{sp}$ is quite
small), then the second term can be neglected, and we have the solution:

\begin{equation}
  \Gamma{g_{th}} = \frac{1}{v_g\tau_p} = <\alpha_i> + \alpha_m
\end{equation}

The product, $\Gamma{g_{th}}$, is referred to as the threshold modal gain
because it now represents the net gain required for the mode as a whole, and it
is the mode as a whole that experiences the cavity loss. $<\alpha_i>$ is the
average internal loss, and $\alpha_m$ is the mirror loss if we considered an
in-plane wave laser.

The optical energy of a nano-cavity laser propagates in a dielectric waveguide
mode, which is confined both transversely and laterally as defined by a
normalized transverse electric field profile, $U(x,y)$. In the axial direction
this mode propagates as $exp^{(-j\beta z)}$, where $\beta$ is the complex
propagation constant, which includes any loss or gain. Thus, the time- and
space-varying electric field can be written as

\begin{equation}
  \xi = \hat{e}_{y}E_{0}U(x,y)e^{j(\omega t- \beta z)}
\end{equation}

where $\hat{e}_y$ is the unit vector indicating TE polarization and $E_0$ is
the magnitude of the field. The complex propagation constant, $\beta$, includes
the incremental transverse modal gain, $<g>_{xy}$ and internal modal loss,
$<\alpha_i>_{xy}$. If we consider a Fabry-Perot laser with the propagating mode
is reflected by end mirrors, and the reflection coefficients are $r_1$ and
$r_2$. respectively. In addition, the mean mirror intensity reflection coefficient, $R = r_1r_2$.

Define the mirror lass as $\alpha_m$

\begin{equation}
  \alpha_m \equiv \frac{1}{L}\ln(\frac{1}{R})
\end{equation}

The photon decay lifetime is given by,

\begin{equation}
  \frac{1}{\tau_p} = \frac{1}{\tau_i} + \frac{1}{\tau_m} = v_g(<\alpha_i> + \alpha_m)
\end{equation}

Thus, we can also write

\begin{equation}
  \Gamma g_{th} = <\alpha_i> + \alpha_m = \frac{1}{v_g\tau_p}
\end{equation}

\section{Threshold Current and Output Power}
\label{sec:threshold_current_and_power_out_versus_current}

We construct together a below-threshold and an above-threshold characteristic
to illustrate the output power versus input current for a normal diode laser.
The first step is to use the below-threshold steady-state carrier rate
equation,

\begin{equation}
  \frac{\eta_{i}I_{th}}{qV} = {(R_{sp} + R_{nr} + R_{l})}_{th} = \frac{N_{th}}{\tau} 
\end{equation}

Then, recognizing that $(R_{sp} + R_{nr} + R_{l}) =AN + BN^2 +CN^3$ depends
monotonically on $N$, we observe from eq 2.34 that above threshold $(R_{sp} +
R_{nr} + R_{l})$ will also clamp at its threshold value, given by Eq 2.35. Thus,
we can substitute Eq 2.35 into the carrier rate equation, eq 2.16 to obtain a
new above threshold carrier rate equations,

\begin{equation}
  \frac{dN}{dt} = \eta_i \frac{(I - I_{th})}{qV} - v_{g}gN_p,~~~   (I > I_{th})
\end{equation}

We also calculate a steady-state photon density above threshold where $g = g_{th}$,

\begin{equation}
  N_p = \frac{\eta_i (I - I_{th})}{qv_{g}g{th}V}~~~   (steady~ state)
\end{equation}

To obtain the power out, we first construct the stored optical energy in the
cavity, $E_{os}$, by multiplying the photon density, $N_p$, by the energy per
photon, $hv$, and the cavity volume, $V_p$. That is $E_{os} = N_phvV_p$. Then,
we multiply this by the erngy loss rate through the mirrors, $v_g\alpha_m =
\frac{1}{\tau_m}$, to get the optical power output from the mirrors,

\begin{equation}
  P_0 = v_g\alpha_{m}N_phvV_p
\end{equation}

Substituting from , and using $\Gamma = V/V_p$,

\begin{equation}
  P_0 = \eta_i(\frac{\alpha_m}{<\alpha_i> + \alpha_m})\frac{hv}{q}(I - I{th}),~~~(I > I_{th})
\end{equation}

For the output power below-threshold $(I < I_{th})$ can be approximated by neglecting the stimulated emission term and solving for $N_p$ under steady-state conditions.

\begin{equation}
N_p = \Gamma\beta_{sp}R_{sp}\tau{p}~~~(I < I_{th})
\end{equation}

and 

\begin{equation}
  P_0(I < I_{th}) = \eta_r\eta_i\left(\frac{\alpha_m}{<\alpha_i> + \alpha_m}\right)\frac{hv}{q}\beta_{sp}I,
\end{equation}


We can get the threshold carrier density

\begin{equation}
  N_{th} = N_{tr}e^{g_{th}/g_{0}N} = N_{tr}e^{(<\alpha_i> + \alpha_m)/\Gamma{g_{0}}N}
\end{equation}

Using the polynomial fit for the recombination rates, and recognizing that for
the best laser material the recombination at threshold is dominated by
spontaneous recombination, we have, $I_{th}\cong B{N_{th}}^2qV/\eta_i$, Thus

\begin{equation}
  I_{th} {\cong} \frac{qVB{N_{tr}}^2}{\eta_i}e^{(<\alpha_i> + \alpha_m)/\Gamma{g_0}N}
\end{equation}

where for most $\uppercase\expandafter{\romannumeral3} -
\uppercase\expandafter{\romannumeral5}$ materials the bimolecular recombination
coefficient, $B \sim 10^{-10} cm^3/s$.

For a multiple quantum-well (MQW) lasers, we have to multiply the single-well confinement factor, $\Gamma_1$, and volume, $V_1$, by the number of wells, $N_w$.


\begin{equation}
  I_{thMQW} {\cong} \frac{qN_{w}V_{1}B{N_{tr}}^2}{\eta_i}e^{2(<\alpha_i> + \alpha_m)/{N_w\Gamma_{1}g_{0}N}}
\end{equation}



