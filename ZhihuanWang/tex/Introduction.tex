\chapter{Introduction}

The practice of staining galss for decorative purposes dates to ancient Rome,
but the investigation of light matter interaction starts even back to the
begining of history for human beings, when ancient lifes first notice the
existence of sun or the first time to use torch for hunting. People never stop
to study how the light interact with our world, meanwhile, implement our
understanding to build a better world with lighting technology. The
stained-glass windows such as the one inside of the St. Patrick's Cathedral in
Fig.~\ref{StainGlass} served as a 'poor man's Bible' in the Middle Ages,
allowing believers who could not read Latin to learn the story of the Gospels.
The term stained glass refers to glass that has been coloured by adding
metallic salts during its manufacture. Then this coloured glass is crafted into
stained glass windows in which small pieces of glass are arranged to form
patterns or pictures. Even nowadays, people are still surprised about how
beautiful they are and wondering how the light interact with these pieces of
art crafts. Besides the arts, lighting technology is also evolving very
rapidly, from the ancient torchs to the bulb that Thomas Edison invented and to
the modern LEDs such as the one in Fig.~\ref{StainGlass}. From balckbody
radiation to electroluminescence, i.e., from thermal radiation to
electrons-holes recombination, we are now capable to generate light more clean
and more efficiently.

\begin{figure}
  \caption{The Stained-glass window and the top modern lamps inside of the St. Patrick's Cathedral, 5th Ave, New York, NY.}
  \centering
  \includegraphics[width=0.5\textwidth,height=0.5\textheight]{pictures/Introduction/StainGlass}
  \label{StainGlass}
\end{figure}

Now we find ourselves living through a new revolution in the age of information
technology, one with consequences every bit as dramatic and likely even more
profound as the data transmission by light. Electrons served us very well for
the recent few decades, until the explosion of data growth nowadays. The storage
and the transmission of data consumed large amount of power and time,
simultaneously. Meeting the energy needs of the commnuication of information,
together with storage and computation form a "grand challenge" of the
information age.

One good example of huge amount of power comsumption by data transmission and
coputation is the data centers, which currently consume 1.5\% of global energy
production, and up to approximately 4\% of U.S. energy produced. Though the
statistics seems small, a 1000 times increase in the volume of data is
predicted by 2025. Google data center alone consumes enough electricity to
power 200,000 homes, since an average Google search or a YouTube video or a
message through Gmail uses 0.3 watt-hours of electricity. Having efficient data
computaion and transmission tools will greatly reduce the total data center
power consumption into a greener number. And the data pipelines of light can
certainly be very helpful in this regime.


\section{Background} \label{sec:intro_BG}

\subsection{Photonics and Optoelectronics}

Photonics involves the generation, control and detection of light waves and
photons, which are particles of light, in free space or in matter.
Optoelectronics is the study and application of effects related to the
interaction of light and electronic signals, and usually considered a sub-field
of photonics. Both photonics and optoelectronics study the light, and explore a
wider variety of wavelengths besides visible lightwave range, from gamma rays
to radio, including X-rays, ultraviolet and infrared light.

The invention and development of solar
cells~\cite{sun2005organic,perlin1999space},
photodetector~\cite{razeghi1996semiconductor,rogalski2002infrared},
modulators~\cite{chen2011broadband,schuller2010plasmonics},
LEDs~\cite{spanggaard2004brief,schubert2005solid,schubert2005light} and
lasers~\cite{chow2012semiconductor,yablonovitch1987inhibited} certainly set the
example of breakthroughs due to the manipulation of photons in thin films and
semiconductor bulk crystals. The continuing success of photonic technologies
relies on the discovery of new optical materials and the miniaturization of
optoelectronic devices that feature better performance, low cost and low power
consumption. For the last few decades, countless efforts in nano-scale
materials and devices research has created a rich collection of nanostructures
where size, shape and composition can be readily controled. Many such
nanostructures exhibit fascinating optical properties that could have
significant impact in the future for photonic technology. 

\subsection{Core-Shell Nanowires} \label{sec:intro_CSNW}

The primary principle for constantly miniaturizing the device is not only about
the size, but also to have better electronic and optical properties. And
this is achieved by quantization, or to be more specific, the confinement of
electrons and photons. At the very dawn of electronics, the idea of using
heterostructures (i.e., the structure with two layers or regions of dissimilar
crystalline semiconductors) has emerged. After Shockley proposed the idea,
Alferov and Kroemer introduced the concept that heterojuctions could possess
high injection efficiencies in comparison with homojunctions, and we know now
which is due to the confinement of carriers~\cite{alferov2000double}. It would be
very difficult today to imagine solid-state physics without semiconductor
heterostructures for both electronc-based and optical-based applications. The
heterostructures and especially double heterostructures, including quantum
wells, nanowires, and quantum dots, are the fundamental building blocks for
current nanoscience research.

Quantum well is a potential well which confines particles to only move freely
in two dimensions in stead of three dimensions, by forcing them to occupy a
planar region. These wells are formed in semiconductors by having a narrower
bandgap material sandwiched between two layers with wider bandgap materials.
Electrons in quantum wells are confined in two dimensions either naturally or
by doping the barrier of a quantum well, thus a two-dimensional electron gas
(2DEG) may be formed at the heterointerface. Not only an increasing of the
density of states, but also a better performance in optoelectronics devices such
as laser diodes, High Electron Mobility Transistors (HEMTs), photodetectors,
and solar cells.

Quantum dots, as another most common nanostructure in semiconductor physics,
exhibit much more enhanced optical properties. They are normally only several
nano-meters in size, and either synthesized or self-assembled into a bulk
solid. As the particles in the quantum dots are confined in three dimensions,
which leave them zero degree of freedom. As a result, the density of states
changed to a delta function compare to a smooth square root dependence that is
found in bulk materials. The narrower peak spectra and larger magnitude of
intensity make them even better candidates in the application of solar cells,
lasers and light emitting diodes (LEDs).

However, since the introduction in the 1990s, another important class of
semiconductor nanostructures has emerged: structures with cross-sections of
tens or hundreds of nano-meters and lengths up to several micro-meters. These
structures are named as 'nanowires'~\cite{xia2003one} different from quantum
dots as they are confined only in two dimensions, thus allowing electrons,
holes or photons to propagate freely along the third dimension. Besides their
own outstanding electro-optical properties, the high-aspect-ratio of these new
semiconductor nanostructures allows for the bridging of the nanoscopic and
macroscopic world. As Dr. Peidong Yang said in their review paper, "This
nano-macro interface is fundamental to the integration of nanoscale building
blocks in electrical or optoelectronic device applications. Conventional
photonic platforms often consist of features with large aspect-ratios such as
interconnects and waveguides, typically with micrometre dimensions. Thus, when
semiconductor nanowires emerged they were immediately recognized as one of the
essential building blocks for nanophotonics."~\cite{Yan:2009hm}

The development of sophisticated nanowires growth
techniques~\cite{hobbs2012semiconductor,wu2001direct}, either
bottom-up~\cite{lu2007nanoelectronics,Huang:2001kv} or
top-down~\cite{park2009top}, has stimulated a large body of new work in
semiconductor nanowires over the last twenty years or so. Previously, the
research activities focus on the growth of higher quality
nanowires~\cite{Yang:2002ts} and the variation of materials. At that time, most
of the nanowires are core-only with ZnO~\cite{Yang:2002ts},
GaAs~\cite{persson2004solid}, Si~\cite{hochbaum2005controlled}, or
Ge~\cite{wu2000germanium}. However, later on, researchers found out that
growing an additional layer of shell can increase quantum yield by passivating
the surface trap states. In addition, the shell provides protection against
environmental changes, photo-oxidative degradation, and provides another route
for modularity. Precise control of the size, shape and composition of both the
core and the shell enable the emission wavelength to be tuned over a wider
range of wavelengths than with either individual semiconductor.

Undoubtedly, much of this interest was further stimulated by the possibility of
novel physics and applications in core-shell nanowires. 

New physical discovered in inversion channels and
heterostructures, and the first heterostructure electronic devices, such as
modulation-doped field-effect transistors and heterojuntion bipolar
transistors, are now being commercially exploited. Linear optical spectroscopic
techniques, such as absorption, luminescnence and modulation spectroscopy, have
for a long time been important tools in understanding the basic physics of
semiconductor materials. Also over the last fifteen years or so, semiconductor
optical and optioelectronic properties have become of increasing technological
importance in their own right. The ever-growing application of semiconductor
diode lasers and related optoelectronic technology in communications and
consumer products has helped to give yet further impetus to research on
semiconductor optical properties. 

The successes of semiconductor optoelectronics and promising physical
mechanisms and novel devices using quantum-confined structures have,
furthermore, enlivened the debate over possible applications of optics for
other functions such as logic and switching in communications and computation.

It is important to emphasize at the outset that quantum confinement and produce
not only quantitative but also qualitative differences in physics from that in
bulk structures, which is of course another major motivation for the interest
in them. There are many examples of these differences. The optical absorption
spectrum breaks up into a series of steps associated with the quantum-confined
electron and hole levels. Excitonic effects become much stronger because of the
quantum confinement, giving clear absorption resonances even at room
temperature. The relative importance of direct Coulomb screening and exchange
effects is quite different in quantum wells (the Coulomb screening is
relatively much weaker), giving very different optical saturation behaviour.


\section{Literature Review} \label{sec:intro_LR}

Since their introduction in the 1990s, semiconductor nanowires have been
extensively studied and much insight has been gained on tuning their electrical
and optical properties. Nanowire related articles have shown a healthy
increase in number published from 2005 to 2016, as Fig.~\ref{ISIPublication}
(blue bars) shows. Article with topics on optical properties of nanowires
comprise a good portion in all the nanowire-related papers published in the
recent decade, showing clear increasing trend in the number of papers on NW
optics or photonics (green bars), presently comprising more than four-fifth of
the nanowire-related articles.

\begin{figure}
  \caption{Article with topics on optical properties of nanowires consist of a large portion of all the nanowire realted papers published from 2005 to 2016. (Source: ISI website, keyword: Nanowire (blue), Nanowire AND optical OR optoelectronic OR photonics (grey))}
  \centering
  \includegraphics[width=\textwidth]{pictures/Introduction/ISIPublication}
  \label{ISIPublication}
\end{figure}

\section{Scope and Organization of the Dissertation}

This thesis is structured as follows. The growth techniques and electro-optical
properties of core-shell nanowires are presented in Chapter~\ref{data}.  After
introducing four different light confinement mechanisms, i.e., Leaky Mode
Resonance, Whispering Gallery Modes, Fabry-Perot Resonant Mode and Helical
Resonance Modes, Chapter~\ref{LM} presents our findings for a generalized
volumetric modes with light management of sub-wavelength cavities.
Chapter~\ref{ED} presents our methods and findings for calculating band-bending
and electronic distribution in both cylindrical and hexagonal core-shell
nanowire by solving Poisson-Schrodinger equations self-consistantly.  In
Chapter~\ref{RM}, we apply the inter-band optical transition rates study to
understand the extremely enhanced optical properties of hexagonal core-shell
nanowires. and rule out three primary factors (overlap integral, oscillator
strength and joint optical density of states) which are strong function of
dimensionality.  The quantum machenical derivation based on perturabtion theory
and Fermi's Golden Rule used in this chapter are outlined in more detail in
Appendix~\ref{ch:rates}. The modeling of lasing threshold based on the optical
transition rates in Chapter~\ref{LT} confirmed that our theoretical
explanation, analysis and calculation of optical properties of core-shell
nanowire have a very strong dependence on electron confimentent. Finally, we
present our conclusions in Chapter~\ref{conclusions}. 

Throughout this work we use the term  {\em nanowires} (NWs) to represent a
specific quantum confined structure with cross-sections of 2-300 nm and lengths
upwards of several micrometers. There are other research groups using terms
such as {\em nanopillars}, {\em nanotubes} or {\em quantum well wires} (QWRs)
to discuss the same nanostructures.
