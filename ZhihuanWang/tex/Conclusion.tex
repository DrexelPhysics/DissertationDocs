\chapter{Conclusions and Future Research} \label{conclusions}

As mentioned previously, communication of information, together with storage
and computation form a “grand challenge” of the information age.  Recently, the
analysis of big data has become the engine for societal, financial, scientific,
and technological endeavors. This demands an infrastructure that is capable of
fast and reliable high volume data processing. Traditionally, this requirement
was fulfilled by silicon technology. However, silicon-based technology has its
own limitations, such as speed limit and heat dissipation problem. In order to
process high volume data, we need data computation, storage and communication
work as three fundamental functions of a computation cell. And core-shell
nanowires will play an important role in this regime with their extraordinary
optical properties.

In conclusion, fabrication techniques, electrical and optical properties of
core-shell nanowires grown on GaAs or Si substrates were discussed here
emphasizing the analysis of resonant optical modes which depend both radially
and axially on the geometries of the nanowires. This shows how such
sub-wavelength structures can form optical cavities as-grown, without needing
sophisticated facet mirrors. In addition, we show how the fortuitous overlap of
the reduced dimensional electronic wave functions and the photonic modes is
responsible for the extraordinary optoelectronic properties of core-shell
nanowires. Such nano-structures have been developed on heterogeneous
substrates, particularly silicon, and as such becoming an important component
in the next generation of photonic integrated circuits which are particularly
useful in meeting the grand challenge of low energy and fast speed computation.

\section{Contributions of this dissertation}

In this dissertation, we designed and fabricated the CSNWs based
heterostructure devices, measured and simulated their opto-electronic
properties, and compared it with bulk structures. The static behavior
simulations, including 2-D potential profile, electric field distributions, and
carrier concentration, were performed with commercially available software. The
light confinement and  distribution in the CSNWs was investigated by FDTD
simulation. The simulation revealed the transverse and longitudinal plane waves
in the resonance frequency enhanced the optical confinement of these sub-micron
scale cavities.

We showed that how low dimensional electron density distribution change the optical transition rates when a small perturbation is introduced by light
which results in large enhancement of optical properties.

In addition, we designed a CSNW based laser and modeled its static and dynamic
lasing behavior by calculating the optical gain and threshold current density,
demonstrated that how reduced dimensional electron states enhance the overall
gain and emission efficiency compared to its bulk counterpart.

The major contributions of this thesis are (1) design, fabricate and
characterize the hexagonal CSNWs grown on Si or GaAs substrates, revealing enhanced 
optical properties, such as absorption, emission and lasing; (2)
simulation and analysis of light confinement and propagation mechanism in
CSNWs; (3) simulation and analysis of electron distribution in the hexagonal
CSNWs; (4) derived dimensional dependent band-to-band transition rates and
proposed the spacial overlapping of the confined light with reduced electronic
wave function greatly enhanced the optical transition rates; (5) modeled and
calculated the optical gain and output power in order to justify that lower
dimensional electron states will facilitate the lasing behaviors of CSNW based
laser and produced more light compare to their bulk counterparts.

\section{Outline of the future work}

%Insert amazing opening paragraph here.  An overview of the goals of this thesis.
%General idea: Take a group of observables, look for correlations, relate to physics.

\subsection{Plasmonic Effect}

Low dimensional electron gases exsit at the heterointerfaces of core-shell
nanowires (CSNWs). For example, the GaAs/AlGaAs CSNWs typically form a
hexagonal structure in which six (6) pillars of 1D charge at the vortices, and
six (6) sheets of 2D charge at facets  are formed~\cite{Wang:2015hz}. At the same time,
nanowires (NW) have also been shown to be capable of confining light in their
sub-wavelength nano-structure, supporting photonic modes, and producing
resonant cavities without the need for polished end facets. We have previously
shown how the electronic wave functions that are thus formed affect the optical
transition rates, resulting in orders of magnitude  enhancement in absorption
and emission of light. Here we propose the plasmonic effects of the
confined charge on the optical properties of CSNWs.

The finite difference time domain (FDTD) simulations identify the
surface plasmon resonance modes which affect light confinement in hexagonal
CSNWs, and help form a high quality factor resonant cavity. We
compare regular CSNW, with a) wires covered with metal which produces surface
plasmon-polaritons (SPP’); b) NWs covered with metal that is sandwiched between
the core and the outer, shell; and c) two-dimensional electron gas (2DEG)
which embedded at the heterointerace of CSNWs. Results show that the 2DEG
behaves similarly to an embedded metallic surface, allowing for highly
localized light confinement in these wires without the need for vertical
structures such as Bragg mirrors commonly used in vertical cavity surface
emitting lasers (VCSEL’s). Besides affecting the cavity, the 2DEG enhances  the
transition rates due to the plasmon-electron interaction, facilitating not only
photonic stimulated emission and lasing, but also  surface plasmon
amplification by stimulated emission of radiation~\cite{Bergman:2003vo}.

The electromagnetic wave traveling of the Surface Plamon Polariton (SPP)
involves both charge motion in electron reservoir (e.g., metal, graphene and
2DEG) and waves in the dielectric or air. Instead of using any metallic
materials, Core-Shell nanowires (CSNWs) can naturally form two-dimensional
electron gas (2DEG) at the heterojunction interface and even large
one-dimensional pillar of charge at the corners of thier hexagonal facets.


\begin{figure}
  \caption{An FDTD-simulated electric field profile (linear scale) of (a) a hexagonal core-shell nanowire (CSNW), (b) photonic modes are affected by plamonic modes in a CSNW covered with silver coating, (c) CSNW with embedded silver layer between the core and the shell; (d) plasmonic and photonic modes of CSNW with embedded 2DEG show similar effects compared to embedded metal. The black boundaries represent the interface betweeen layers of the structure.}
  \centering
  \includegraphics[width=\textwidth]{pictures/Conclusion/PlasmonMode}
  \label{PlasmonMode}
\end{figure}

Figure~\ref{PlasmonMode} shows the FDTD-simulated electric field profile
(linear scale) in the transverse plane of (a) CSNW; (b) CSNW with silver
coating; (c) CSNW with and embedded silver layer between the core and the
shell; (d) CSNW with 2DEG at the hetero-interface. As shown in
Fig~\ref{PlasmonMode}, coating the wire with metal introduces plasmonic modes
in the structure that enhance light confinement. Metal embedded between the
core and the shell has similar effect. Importantly, we observe that similar
plasmonic features can be obtained due to the 2DEG that is embedded in
CSNW~\cite{montazeri2016plasmonic}.

\subsection{Monolithic Integration on PIC}

Recently, the increasing demand for high-speed low-power computation and
communication has driven the growth of photonics integrated circuit
(PIC) technology with projected market size of a billion dollars by 2018
. Silicon photonics have received significant attention as it
benefits from well-established complementary metal-oxide-semiconductor
(CMOS) technology. Lack of an efficient silicon-based light source and
photodetector, has motivated development of technologies for
heterogeneous integration of efficient III-V semiconductor light sources
and detector with silicon chips. Here we present core-shell
nanowires (CSNWs) as versatile low-dimensional optoelectronic systems as
a replacement to their conventional thin film counterparts in
heterogeneous integration in silicon photonics. These CSNWs have
extraordinary performance in light generation, absorption, light
modulation, energy generation, and high-speed optical detection.
Finally, we elaborate on a vision for a low-cost high-performance
silicon photonics chip based on a core-shell nanowire platform. In this
scheme, CSNWs are applied as high-speed low-power optical detectors,
light source, and waveguides.

In order to process high volume data,
we need data computation, storage, and communication to work in concert as the
three fundamental functions of a computation cell. As schematically shown in
Fig.~\ref{NWPIC_NB}, a monolithic nanosystem may be envisioned, which
incorporates NWs as waveguides, detectors, photovoltaic cells, antennas,
modulators, (photo)capacitors, LEDs and lasers. These components may be
incorporated in circuit layers, such as network-on-chip. Different layers can
communicate using NW through-silicon vias (TSVs). Similar
low-power/high-performance advantags can be realized through achievement of
high interconnect densities on the 2.5D through-Si-interposer (TSI) as reported
in reference~\cite{Zhang:2015ec} 

Performance
enhancement of CSNWs can be attributed to the
formation of confined quasi-one-dimensional electron gas at the
vortices, and two-dimensional electron gas at the facets of the
hexagonally shaped GaAs/AlGaAs core-shell hetero-interface.
These reduced dimensional confined charge plasma affect optical
transition rates, facilitate population inversion, and collect optically
generated electrons and wholes before they transit to the contacts.
Additionally these plasmons, which are unique to CSNWs, affect
waveguiding and optical cavity properties of CSNWs and can be used as
waveguides, modulators and photocapacitors.

The proposed integrated photonic platform is schematically depicted at
the bottom part of Fig.~\ref{NWPIC_NB} with multiple key components implemented
through utilizing core-shell nanowires. This monolithic nanosystem
incorporates CSNWs as waveguides, detectors, photovoltaic cells,
antennas, modulators, LEDs, and lasers. These components may be
incorporated in circuit layers, such as networks on chip. Additionally,
they can be used for 3D integration using NW through-silicon vias
(TSVs). Such a circuit may compete with manufacturing methods such as
flip-chip bonding and achieves further miniaturization by incorporating
high-performance nanowires in vertical architectures to replace large
surface area thin film structures such as vertical cavity surface
emitting lasers (VCSELs).

In conclusion, CSNW demonstrate unique combination of plasmonic, photonic,
and electronic properties which makes them versatile high-performance
optoelectronic devices including Lasers, LEDs, photodetectors, solar
cells, waveguides, and optical amplifiers. Since they can be grown from
a wide range of material including GaAs, InP, and GaN, at different
directions and on foreign substrates such as oxides, Graphene, Si, and
III-Vs, they offer a competitive platform for photonic integrated
circuits, and specifically for heterogeneous integration in silicon
photonics chips, photodetectors/photocapacitors, antennas and
waveguides.

\begin{figure}
  \caption{Schematic depiction of an optoelectronic nanosystem may include key components such as NW LED/laser source, photodetector/photocapacitor, NW antennas, and NW-enabled network-on-chip integrated on silicon.}
  \centering
  \includegraphics[width=\textwidth]{pictures/Conclusion/NWPIC_NB}
  \label{NWPIC_NB}
\end{figure}
%In Chapter~\ref{data} we brought together data from the mid-IR through the UV
%for the purpose of creating multi-wavelength SEDs for the SDSS DR7 quasars
%catalog.  This involved cross-matching mid-IR data from {\em Spitzer} and {\em
%WISE}, near-IR data from 2MASS and UKIDSS, and UV data from {\em GALEX}.  From
%this cross matched data set we created several subsets used throughout our
%studies.  We began with a (observationally) non-reddened data set used to
%explore trends in the SEDs based on various observed properties. To study the
%dust reddening properties of our quasars we limited our data to quasars
%uniformly selected by the SDSS quasar detection pipeline.  This data set was
%further split into quasar with BALs and quasars without BALs.  Finally, when
%exploring our SEDs as function of $\mbh$ we further limited this sample to the
%non-BAL quasars since BALs can make the estimated values for $\mbh$ invalid.

