\chapter{Conclusions} \label{conclusions}

%Insert amazing opening paragraph here.  An overview of the goals of this thesis.
%General idea: Take a group of observables, look for correlations, relate to physics.

%Throughout this thesis we have explored the various properties of quasar SEDs.  In all cases we have taken a group of observables, look for correlations, and related these correlations back to the physical processes of the quasars. Here we  summarize the data used and the major findings and physical interpretations from each chapter.

The physics that drives the accretion of matter onto the central black hole (in its simplest form) only relies on three parameters: $\mbh$, $\dot{M}_{\rm BH}$, and the spin of the central black hole.  As such, constructing SEDs based on these quantities would provide the most insight into the physics of the quasar.  Both $\mbh$ and $\dot{M}_{\rm BH}$ can be estimated from the data, but to do so accurate {\em intrinsic} \lbol\ values must be obtained for each quasars in our sample.   

%In Chapter~\ref{data} we brought together data from the mid-IR through the UV for the purpose of creating multi-wavelength SEDs for the SDSS DR7 quasars catalog.  This involved cross-matching mid-IR data from {\em Spitzer} and {\em WISE}, near-IR data from 2MASS and UKIDSS, and UV data from {\em GALEX}.  From this cross matched data set we created several subsets used throughout our studies.  We began with a (observationally) non-reddened data set used to explore trends in the SEDs based on various observed properties. To study the dust reddening properties of our quasars we limited our data to quasars uniformly selected by the SDSS quasar detection pipeline.  This data set was further split into quasar with BALs and quasars without BALs.  Finally, when exploring our SEDs as function of $\mbh$ we further limited this sample to the non-BAL quasars since BALs can make the estimated values for $\mbh$ invalid.

Before looking for trends in quasar SEDs based on estimated quantities, in Chapter~\ref{SEDs} we explored  changes in our SEDs based our various observational properties.
%One of our goals in Chapter~\ref{SEDs} was to explore changes in our SEDs based our various observational properties.  
In particular, we looked at the observed luminosity (\ltwofive) and properties of the \civ\ emission line. 
We created three luminosity-dependent mean SEDs and found the high-luminosity quasars had bluer optical continua and more hot dust emission than low-luminosity quasars. Additionally, we found $\alpha_{{\rm FUV}}$ to be dependent on luminosity in the sense that more luminous quasars have redder $\alpha_{{\rm FUV}}$ continua, consistent with \citet{Scott:2004}. 
When looking at the \civ\ properties, we saw differences in the Balmer continuum, Ly$\alpha$ and \civ\ (by construction) line strengths that are consistent with eigenvector 1 trends \citep{Brotherton:1999}.

In addition to these mean SEDs, we also constructed individual SEDs and found bolometric corrections for each quasars.  We found that the more significant contribution to the \bctwofive\ distribution was not the observed IR--near-UV parts of the SED, but instead the unseen EUV part of the SED.  Given these findings, it is important to consider potentially significant differences in the EUV part of the SED at the extrema of the quasars population.

One of the biggest secondary effects that change the shape of the observed SED in the optical--UV (and hence \lbol) is dust extinction associated with the quasar.  In Chapter~\ref{Dust} we used our mid-IR--UV photometric data to calculate the intrinsic colors and amount of extinction in each quasar assuming various dust reddening laws.  Using a hierarchical Bayesian model we broke the degeneracy typically seen between these variables, and bounded them within a physical range.  The majority of our data were well fit by an SMC reddening law, but a small set were just as well fit by the steeper multi-scattering law from \citet{Leighly:2014}.  From the individual fits we found the BAL quasars tend to be biased towards the blue end of the parent population of colors and have more extinction as compared to the non-BAL quasars,
%slightly bluer and have more extinction then the non-BAL quasars, 
and in general the BAL fraction is very dependent on the intrinsic color and amount of extinction in the quasar sample.

Looking at composite spectra as functions of extinction and intrinsic color revealed the following trends in both the BAL and non-BAL samples.  On average, we found the fit values based on the photometry do a good job of reproducing the slopes and curvatures seen in the composite spectra.  With the BAL spectra we also found the width and depth of the \civ\ absorption troughs changed with both color and extinction.

To look for trends in the spectra based on a quasar's intrinsic color, we compared the spectra in the lowest reddening bin.  In general, the intrinsically red quasars have stronger ionized spectral lines (e.g. \ion{C}{4}, \ion{He}{2}, \ion{C}{3}], and [\ion{O}{2}]), while the bluer quasars have stronger Balmer lines (e.g. H$\gamma$, H$\beta$, H$\alpha$, and the SBB).  These trends are consistent with the redder quasars having a harder SED in the EUV than the bluer quasars.  Additionally, the [\ion{O}{2}] line may indicate that the host galaxy of the red quasars is underestimated by our simple model from Chapter~\ref{SEDs}.  In the BAL spectra, similar trends were found in the emission lines, and additionally, the absorption troughs respond to changes in the underlying SED such that the redder quasars had stronger and lower velocity absorption troughs as compared to the bluer quasars.

In Chapter~\ref{BH} we extended the analysis of Chapter~\ref{Dust} by constructing full SEDs as function of color and extinction.
Using the SMC extinction values, the effects of dust reddening were removed from the SEDs, allowing us to study the intrinsic SEDs.  From these SEDs, we found trends with both color and extinction.  
Although the cause is unclear, we found evidence for either an observational bias or intrinsic effect that changes the SEDs as a function of $\ebv$.
Additionally, the intrinsically bluer quasars have more hot dust, a BBB peaking at shorter wavelengths, and higher \bctwofive.  
In the EUV, the blue SEDs are harder than the spectra from Chapter~\ref{Dust} would suggest.
In order to reconcile the SEDs with the spectra, we conclude that the BLR sees a very different continuum than we do (e.g., orientation effects or shielding).  
%These trends are consistent with a hotter accretion disk and/or accretions disks being seen closer to edge-on.

These (internal) extinction corrected SEDs allowed us to shift our analysis to the more physical properties of $\mbh$ and accretion rate (estimated with $\lf$).  
As predicted by the $\alpha$-disk theory of \citet{Shakura:1973}, we found the quasars with high accretion rates have SEDs consistent with hotter accretion disks.  As expected, these quasars were also more luminous and had higher \bctwofive.  Contrary to the $\alpha$-disk theory, the quasars with large $\mbh$ were also consistent with a hotter accretion disk.  These SEDs were also slightly more luminous then the SEDs with smaller $\mbh$, and there were no significant trends in \bctwofive.

With the advent of large digital surveys and sophisticated selection algorithms, the number of detected quasars has grown from a few hundred to over one million in the past ten years alone \citep{Richards:2009}.  As this number has grown it has become apparent that a quasar's SED is not universal, and it is dangerous to assume the results derived from a few hundred of the brightest quasars can be applied to the whole population.  Throughout this work we have explored both the individual and population parameters for a large sample of quasars, providing the astronomical community with a catalog that can be used study the physical properties of quasars without having to rely on scaling relations calibrated on a subset of quasars.