\chapter{Dimensional Dependence of Optical Transition Rates} \label{RM}
 
\section{Time-dependent Perturbation Theory} \label{dust_seds}

\subsection{} \label{dust_corrections}

%\subsection{A Uniform Sample}
%\subsection{\ltwofive\ and \bctwofive\ Distributions} \label{l_bc_dust}
%\subsection{Extinction Corrected SEDs} \label{l_bc_dust}

\section{Upward and Downward Transition Rates} \label{mbh_seds}

\section{Contributing Factors} \label{BH_conclusions}
\subsection{Overlap Integral}
\subsection{Oscillator Strength}
\subsection{Joint Optical Density of States}
The preceding theory of gain involving Fermi's Golden Rule considers each electron in isolation as it interacts with the electromagnetic field. In other words, we have used a single-particle theory to obtain the gain spectrum. In reality, there is a large density of both electrons and holes present in the system. The mutual interactions between these particles are generally referred to as many-body effects. These effects included lineshape broadening, which is related to collisions between particles and/or phonons in the crystal. In addition to this important effect, there are two other significant consequences of many-body effects: exciton states and bandgap shrinkage. Exciton states exsit primarily at low carrier densityies and low temperatures, where bandgap shrinkage becomes noticeable at high carrier densities.

Under conditions of low carrier density and low temperature, it is possible for an electron and hole to orbit each othere for an extended period of time, forming what is referred to as an exciton pair. Such exciton pairs have a binding energy associated with them that is euqal to the energy required to separate the electron and hole. As a rsult, electrons that are elevated from the valence band to one of these exciton states will absorb radiation at energies equal to the bandgap less the binding energy (the bandgap will appear to be red-shifted). More significatnly however, the overlap integral ( and hence the matrix element ) of these two-particle states can be quite large. As a result, band-to-exciton transitions tend to dominate the absorption spectrum. However, exciton states are limited to states near $k = 0$, and hence band-to-exciton transitions are clustered at the band edge (or subbabdn edge). The overall effect is the qppearance of very strong absorption peaks near the subband edges in quantum-well materials, and near the band edge in bulk material.
Exciton absorption peaks are clearly visible in quantum wells at room temperature for a typical GaAs QW. The first two steps in the "staircase" absorption spectrum predicted from the density of states. However, the exciton peaks riding on top of the steps, particularly the n = 1 peaks, dominate the absorption spectrum. Each observed exciton peak corresponds to one of the subband transitions.

The second many-body effect occurs at high carrier densities, where the charges actually screen out the atomic attactive forces. With a weaker effective atomic potential, the single-atom electron wavefunctions of interest become less localized and the nearest-neighbor electron everlap becomes higher.  The large overlap increases the width of the energy bands ($\delta{E}$ is larger), reducing the gap between bands. Although this description is only qualitative, it does reveal that the bandgap should shrink with increasing carrier density.
It can also be argued theoretically that the badgap shrinkage is inversely related to the average spacing between carriers, or (the closer the carriers are, the more their own Coulomb potentials screen out the atomic potential). In bulk material, the average volume occupied by one carrier is inversely related to the carrier density. 
The net effect of bandgap shrinkage is that as carrier density increases, the entire gain spectrum redshifts by a noticeable amount. In principle, the shift is accompanied by a slight distortion. (i.e, reshaping and enhancement) of the spectrum.
